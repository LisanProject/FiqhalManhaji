% Options for packages loaded elsewhere
% Options for packages loaded elsewhere
\PassOptionsToPackage{unicode}{hyperref}
\PassOptionsToPackage{hyphens}{url}
\PassOptionsToPackage{dvipsnames,svgnames,x11names}{xcolor}
%
\documentclass[
  a4paper,
  DIV=11,
  numbers=noendperiod]{scrartcl}
\usepackage{xcolor}
\usepackage{amsmath,amssymb}
\setcounter{secnumdepth}{5}
\usepackage{iftex}
\ifPDFTeX
  \usepackage[T1]{fontenc}
  \usepackage[utf8]{inputenc}
  \usepackage{textcomp} % provide euro and other symbols
\else % if luatex or xetex
  \usepackage{unicode-math} % this also loads fontspec
  \defaultfontfeatures{Scale=MatchLowercase}
  \defaultfontfeatures[\rmfamily]{Ligatures=TeX,Scale=1}
\fi
\usepackage{lmodern}
\ifPDFTeX\else
  % xetex/luatex font selection
\fi
% Use upquote if available, for straight quotes in verbatim environments
\IfFileExists{upquote.sty}{\usepackage{upquote}}{}
\IfFileExists{microtype.sty}{% use microtype if available
  \usepackage[]{microtype}
  \UseMicrotypeSet[protrusion]{basicmath} % disable protrusion for tt fonts
}{}
\makeatletter
\@ifundefined{KOMAClassName}{% if non-KOMA class
  \IfFileExists{parskip.sty}{%
    \usepackage{parskip}
  }{% else
    \setlength{\parindent}{0pt}
    \setlength{\parskip}{6pt plus 2pt minus 1pt}}
}{% if KOMA class
  \KOMAoptions{parskip=half}}
\makeatother
% Make \paragraph and \subparagraph free-standing
\makeatletter
\ifx\paragraph\undefined\else
  \let\oldparagraph\paragraph
  \renewcommand{\paragraph}{
    \@ifstar
      \xxxParagraphStar
      \xxxParagraphNoStar
  }
  \newcommand{\xxxParagraphStar}[1]{\oldparagraph*{#1}\mbox{}}
  \newcommand{\xxxParagraphNoStar}[1]{\oldparagraph{#1}\mbox{}}
\fi
\ifx\subparagraph\undefined\else
  \let\oldsubparagraph\subparagraph
  \renewcommand{\subparagraph}{
    \@ifstar
      \xxxSubParagraphStar
      \xxxSubParagraphNoStar
  }
  \newcommand{\xxxSubParagraphStar}[1]{\oldsubparagraph*{#1}\mbox{}}
  \newcommand{\xxxSubParagraphNoStar}[1]{\oldsubparagraph{#1}\mbox{}}
\fi
\makeatother


\usepackage{longtable,booktabs,array}
\usepackage{calc} % for calculating minipage widths
% Correct order of tables after \paragraph or \subparagraph
\usepackage{etoolbox}
\makeatletter
\patchcmd\longtable{\par}{\if@noskipsec\mbox{}\fi\par}{}{}
\makeatother
% Allow footnotes in longtable head/foot
\IfFileExists{footnotehyper.sty}{\usepackage{footnotehyper}}{\usepackage{footnote}}
\makesavenoteenv{longtable}
\usepackage{graphicx}
\makeatletter
\newsavebox\pandoc@box
\newcommand*\pandocbounded[1]{% scales image to fit in text height/width
  \sbox\pandoc@box{#1}%
  \Gscale@div\@tempa{\textheight}{\dimexpr\ht\pandoc@box+\dp\pandoc@box\relax}%
  \Gscale@div\@tempb{\linewidth}{\wd\pandoc@box}%
  \ifdim\@tempb\p@<\@tempa\p@\let\@tempa\@tempb\fi% select the smaller of both
  \ifdim\@tempa\p@<\p@\scalebox{\@tempa}{\usebox\pandoc@box}%
  \else\usebox{\pandoc@box}%
  \fi%
}
% Set default figure placement to htbp
\def\fps@figure{htbp}
\makeatother





\setlength{\emergencystretch}{3em} % prevent overfull lines

\providecommand{\tightlist}{%
  \setlength{\itemsep}{0pt}\setlength{\parskip}{0pt}}



 


\KOMAoption{captions}{tableheading}
\makeatletter
\@ifpackageloaded{tcolorbox}{}{\usepackage[skins,breakable]{tcolorbox}}
\@ifpackageloaded{fontawesome5}{}{\usepackage{fontawesome5}}
\definecolor{quarto-callout-color}{HTML}{909090}
\definecolor{quarto-callout-note-color}{HTML}{0758E5}
\definecolor{quarto-callout-important-color}{HTML}{CC1914}
\definecolor{quarto-callout-warning-color}{HTML}{EB9113}
\definecolor{quarto-callout-tip-color}{HTML}{00A047}
\definecolor{quarto-callout-caution-color}{HTML}{FC5300}
\definecolor{quarto-callout-color-frame}{HTML}{acacac}
\definecolor{quarto-callout-note-color-frame}{HTML}{4582ec}
\definecolor{quarto-callout-important-color-frame}{HTML}{d9534f}
\definecolor{quarto-callout-warning-color-frame}{HTML}{f0ad4e}
\definecolor{quarto-callout-tip-color-frame}{HTML}{02b875}
\definecolor{quarto-callout-caution-color-frame}{HTML}{fd7e14}
\makeatother
\makeatletter
\@ifpackageloaded{caption}{}{\usepackage{caption}}
\AtBeginDocument{%
\ifdefined\contentsname
  \renewcommand*\contentsname{Table of contents}
\else
  \newcommand\contentsname{Table of contents}
\fi
\ifdefined\listfigurename
  \renewcommand*\listfigurename{List of Figures}
\else
  \newcommand\listfigurename{List of Figures}
\fi
\ifdefined\listtablename
  \renewcommand*\listtablename{List of Tables}
\else
  \newcommand\listtablename{List of Tables}
\fi
\ifdefined\figurename
  \renewcommand*\figurename{Figure}
\else
  \newcommand\figurename{Figure}
\fi
\ifdefined\tablename
  \renewcommand*\tablename{Table}
\else
  \newcommand\tablename{Table}
\fi
}
\@ifpackageloaded{float}{}{\usepackage{float}}
\floatstyle{ruled}
\@ifundefined{c@chapter}{\newfloat{codelisting}{h}{lop}}{\newfloat{codelisting}{h}{lop}[chapter]}
\floatname{codelisting}{Listing}
\newcommand*\listoflistings{\listof{codelisting}{List of Listings}}
\makeatother
\makeatletter
\makeatother
\makeatletter
\@ifpackageloaded{caption}{}{\usepackage{caption}}
\@ifpackageloaded{subcaption}{}{\usepackage{subcaption}}
\makeatother
\usepackage{bookmark}
\IfFileExists{xurl.sty}{\usepackage{xurl}}{} % add URL line breaks if available
\urlstyle{same}
\hypersetup{
  pdftitle={Introduction to the Shafi'i Madhhab- Taqrirat al-Sadidah},
  pdfauthor={Liban Hussein},
  colorlinks=true,
  linkcolor={blue},
  filecolor={Maroon},
  citecolor={Blue},
  urlcolor={Blue},
  pdfcreator={LaTeX via pandoc}}


\title{Introduction to the Shafi'i Madhhab- Taqrirat al-Sadidah}
\author{Liban Hussein}
\date{2025-05-06}
\begin{document}
\maketitle

\renewcommand*\contentsname{Table of contents}
{
\hypersetup{linkcolor=}
\setcounter{tocdepth}{3}
\tableofcontents
}

\section{Introduction}\label{sec-intro}

The long-term goal of this project is to provide an accessible,
organized study companion for English-speaking students of sacred law,
rooted in reliable Shafi'i methodology and sources (as taught by
\textbf{Shaykh Mahdi Lock})

\begin{center}\rule{0.5\linewidth}{0.5pt}\end{center}

\subsection{Organization}\label{organization}

The content follows the topical arrangement of the original
\emph{Al-Fiqh al-Manhaji} volumes, beginning with purification and
prayer, and progressing through the major sections of worship,
transactions, personal law, and social obligations. Each section is
broken into:

\begin{itemize}
\tightlist
\item
  Main rulings and definitions\\
\item
  Evidence when included\\
\item
  Summary tables where beneficial\\
\item
  Notes, commentary, or clarifications based on trusted works like
  \emph{Taqrirat al-Sadidah}
\end{itemize}

\begin{center}\rule{0.5\linewidth}{0.5pt}\end{center}

\subsubsection{Chapters Currently in
Progress}\label{chapters-currently-in-progress}

\begin{enumerate}
\def\labelenumi{\arabic{enumi}.}
\tightlist
\item
  \textbf{Purification (al-Taharah)}\\
\item
  \textbf{Prayer (al-Salah)}\\
\item
  \textbf{Zakat}\\
\item
  \textbf{Fasting (al-Siyam)}\\
\item
  \textbf{Hajj and Umrah}
\end{enumerate}

More will be added over time, and each chapter will eventually include
details on the footnotes and commentaries by the authors. Before going
into Fiqh al-Manhaji itself, an introduction to the foundation and
origin of the Madhab through a translation of the beginning portion of
Taqrirat al-Sadidah is provided.

\begin{center}\rule{0.5\linewidth}{0.5pt}\end{center}

\begin{tcolorbox}[enhanced jigsaw, leftrule=.75mm, bottomtitle=1mm, colframe=quarto-callout-important-color-frame, colback=white, arc=.35mm, breakable, titlerule=0mm, opacitybacktitle=0.6, opacityback=0, colbacktitle=quarto-callout-important-color!10!white, toptitle=1mm, title=\textcolor{quarto-callout-important-color}{\faExclamation}\hspace{0.5em}{Note}, rightrule=.15mm, toprule=.15mm, left=2mm, coltitle=black, bottomrule=.15mm]

This is a living document and will be updated continuously as lectures
continue, new insights emerge, and the annotations grow. Feedback,
questions, and contributions are welcome.

\end{tcolorbox}

\section{Introduction to the Shafi'i
School}\label{introduction-to-the-shafii-school}

\subsection{The Founder of the School}\label{the-founder-of-the-school}

Muhammad ibn Idris al-Shafi'i al-Muttalibi is the founder of the Shafi'i
school. His lineage connects with that of the Prophet Muhammad ﷺ at
their common ancestor, Abd Manaf. He was born in Gaza in the year 150 AH
and was brought to Mecca, where he began his pursuit of knowledge.

He studied under Imam Khalid ibn Muslim al-Zanji, the Mufti of Mecca, as
well as under al-Fudayl ibn Iyad, Sufyan ibn Uyaynah, and others.

Then he traveled to Madinah at the age of twelve, and memorized the
entire \emph{Muwattaʼ} in nine days, to closely accompany Imam Malik. He
studied under Imam Malik until he became one of his top students. He
also took knowledge from the scholars of Madinah and Mecca and became
qualified to give Fatawa at the age of 15. He engaged deeply in Arabic
linguistics and poetry and was praised for his mastery. Even al-Asma'i
(a transmitter of Arab poetry) benefited from him, and he learned from
the ancient poetry of the Banu Hudhayl tribe.

He traveled to Yemen and took knowledge from scholars such as Mutarrif
ibn Mazin, Hisham ibn Yusuf the judge, Amr ibn Abi Salamah, and Yahya
ibn Hasan. Then he traveled to Iraq and studied with Waki' ibn Jarrah,
Muhammad ibn al-Hasan al-Shaybani, the jurist of Iraq.

He composed the book \emph{Al-Hujjah} and established what became known
as the old school of thought. Major scholars took knowledge from him,
including Imam Ahmad ibn Hanbal, Abu Thawr, and others.\\
He turned towards Egypt, revising many of his earlier legal opinions.
There, he founded his new legal school.

He is regarded as the renewer (mujaddid) of the second century, as he
combined between the sciences of hadith and the intellectual reasoning
of opinion-based jurisprudence. He laid down the foundations of usul
al-fiqh, authoring \emph{Al-Risalah}, the first book on legal theory.
His knowledge extended across hadith, Qur'an, Arabic, grammar, history,
poetry, and theology. He lived a life of asceticism and worship, turning
away from the world and preferring the hereafter. He passed away in
Egypt in the year 204 AH.

Imam Ahmad said: ``Imam al-Shafi'i was like the sun for the world and
like health for the body --- can either of these be replaced or
substituted?''

He also said: ``Knowledge of jurisprudence was locked up with its people
until Allah opened it through al-Shafi'i.''

Imam Abu Zur'ah said about him: ``I do not know of anyone who had a
greater impact on the people of Islam than al-Shafi'i.''

May God have mercy on them and be pleased with them all.

\begin{tcolorbox}[enhanced jigsaw, leftrule=.75mm, bottomtitle=1mm, colframe=quarto-callout-note-color-frame, colback=white, arc=.35mm, breakable, titlerule=0mm, opacitybacktitle=0.6, opacityback=0, colbacktitle=quarto-callout-note-color!10!white, toptitle=1mm, title=\textcolor{quarto-callout-note-color}{\faInfo}\hspace{0.5em}{The Humility of the Imam and the Wisdom of His Legacy}, rightrule=.15mm, toprule=.15mm, left=2mm, coltitle=black, bottomrule=.15mm]

Imam al-Shafi'i once said:

\begin{quote}
``I wish that the people would learn this knowledge and none of it be
attributed to me.''
\end{quote}

Ironically, or perhaps providentially, his du'a was accepted. The
enduring beauty of the Shafi'i school is that it was polished, codified,
and brought to legal maturity by the hands of scholars after him ---
such as \textbf{al-Ghazali}, \textbf{al-Rafi'i}, and \textbf{al-Nawawi}
--- who preserved the madhab in the clearest possible form.

\end{tcolorbox}

\subsection{The Imams of the School}\label{the-imams-of-the-school}

\textbf{Core Students from the 3rd Century:}\\
\textbf{Al-Muzani}, \textbf{al-Buwayti}, \textbf{al-Rabi' al-Muradi},
\textbf{Harmalah}, \textbf{al-Rabi' al-Jizi}, \textbf{Yunus ibn Abd
al-Ala}

\subsubsection{3rd Century}\label{rd-century}

\textbf{Transmitted the old madhhab:}\\
\textbf{Ahmad ibn Hanbal}, \textbf{Abu Thawr}, \textbf{al-Zafarani},
\textbf{al-Karabisi}

\begin{center}\rule{0.5\linewidth}{0.5pt}\end{center}

\subsubsection{4th Century}\label{th-century}

\textbf{Ibn Surayj}, \textbf{al-Qaffal al-Kabir al-Shashi}, \textbf{Abu
Hamid al-Isfarayini}, \textbf{al-Ishtakhri}, \textbf{al-Marwazi},
\textbf{Ibn Abi Hurayrah}, \textbf{Ibn al-Qass}

\begin{center}\rule{0.5\linewidth}{0.5pt}\end{center}

\subsubsection{5th Century}\label{th-century-1}

\textbf{al-Mawardi}, \textbf{Abu Ishaq al-Shirazi}, \textbf{Abu Muhammad
al-Juwayni}, \textbf{Imam al-Haramayn}, \textbf{al-Bayhaqi},
\textbf{al-Bandaniji}, \textbf{al-Muhamali},\\
\textbf{al-Qaffal al-Saghir al-Marwazi}, \textbf{al-Qadi Husayn},
\textbf{al-Furani}, \textbf{al-Masudi}, \textbf{Ibn al-Sabbagh},
\textbf{al-Mutawalli}

\begin{center}\rule{0.5\linewidth}{0.5pt}\end{center}

\subsubsection{6th Century}\label{th-century-2}

\textbf{al-Ghazali (Hujjat al-Islam)}, \textbf{al-Shashi},
\textbf{al-Baghawi}, \textbf{al-Imrani}

\begin{center}\rule{0.5\linewidth}{0.5pt}\end{center}

\subsubsection{7th Century}\label{th-century-3}

\textbf{Ibn al-Salah}, \textbf{al-Qazwini}, \textbf{al-Izz ibn Abd
al-Salam}, \textbf{al-Nawawi}, \textbf{al-Rafi'i}, \textbf{Ibn
al-Firkah}, \textbf{Ibn Daqiq al-Id}

\begin{center}\rule{0.5\linewidth}{0.5pt}\end{center}

\subsubsection{8th Century}\label{th-century-4}

\textbf{Ibn al-Rif'ah}, \textbf{al-Taqi al-Subki}, \textbf{al-Qamuli},
\textbf{al-Isnawi}, \textbf{al-Adhra'i}, \textbf{al-Bulqini},
\textbf{Ibn al-Mulaqqin},\\
\textbf{al-Zarkashi}, \textbf{Ibn al-Naqib}, \textbf{al-Sharif
al-Barizi}, \textbf{al-Muhibb al-Tabari}

\begin{center}\rule{0.5\linewidth}{0.5pt}\end{center}

\subsubsection{9th Century}\label{th-century-5}

\textbf{al-Wali al-Iraqi}, \textbf{al-Taqi al-Hisni}, \textbf{al-Shihab
ibn Raslan}, \textbf{Ibn Qadi Shuhbah}, \textbf{Ibn al-Muzajjad},
\textbf{al-Damiri},\\
\textbf{al-Jalal al-Mahalli}, \textbf{al-Afqahisi}, \textbf{Ibn
al-Muqri}, \textbf{Abd Allah ibn Abd al-Rahman Ba-Fadl}

\begin{center}\rule{0.5\linewidth}{0.5pt}\end{center}

\subsubsection{10th Century}\label{th-century-6}

\textbf{al-Jalal al-Suyuti}, \textbf{Zakariyya al-Ansari},
\textbf{al-Khatib al-Shirbini}, \textbf{al-Shihab al-Ramli},
\textbf{al-Shams al-Ramli},\\
\textbf{Ibn Hajar al-Haytami}, \textbf{Abd Allah ibn Umar Ba-Makhrama},
\textbf{Ibn Qasim al-Abbadi}, \textbf{Ba Qushayr}, \textbf{Ibn Ziyad}

\begin{center}\rule{0.5\linewidth}{0.5pt}\end{center}

\subsubsection{11th Century}\label{th-century-7}

\textbf{Burhan al-Birmawi}, \textbf{Ali al-Shabaramilsi}, and others

\begin{center}\rule{0.5\linewidth}{0.5pt}\end{center}

\subsubsection{12th Century}\label{th-century-8}

\textbf{al-Bajuri}, \textbf{al-Sharqawi}, \textbf{al-Bujayrimi},
\textbf{Abd Allah ibn Husayn Bal Faqih}, \textbf{Abd Allah ibn Ahmad
Basudan},\\
\textbf{Sa'id ibn Muhammad Ba'ishin}, \textbf{Abd al-Rahman ibn Sulayman
al-Ahdal}, \textbf{Ali Basabrin}, and others

\begin{center}\rule{0.5\linewidth}{0.5pt}\end{center}

\subsubsection{13th Century}\label{th-century-9}

\textbf{Sayyid Alawi ibn Ahmad al-Saqqaf}, \textbf{Ahmad ibn Zayni
Dahlan}, \textbf{Bakri Shatta}, \textbf{Abd al-Rahman al-Mashhur},\\
\textbf{Abu Bakr ibn Abd al-Rahman ibn Shihab}, \textbf{Abu Bakr ibn
Ahmad al-Khatib}, \textbf{Abd Allah Ba Jammah}, \textbf{Abd Allah ibn
Umar al-Shatiri},\\
\textbf{Ahmad ibn Umar al-Shatiri}, \textbf{Abd al-Rahman ibn Abd Allah
al-Saqqaf}, \textbf{Muhammad ibn Hadi al-Saqqaf}, \textbf{Muhammad ibn
Salim ibn Hafiz}, and others

\begin{center}\rule{0.5\linewidth}{0.5pt}\end{center}

\subsubsection{14th Century and Later}\label{th-century-and-later}

A large number of scholars appeared and authored works in the later
centuries. May Allah be pleased with them all. The scholars of the
madhhab wrote prolifically --- so much so that it is difficult to even
count all their writings.

This is exemplified in the work of \textbf{Imam al-Subki}, \emph{Tabaqat
al-Shafi'iyyah al-Kubra}, which spans ten volumes. It demonstrates the
vast number of scholars in the Shafi'i tradition, the range of their
service, and how nearly every field of Islamic knowledge has Shafi'i
representatives.

\begin{center}\rule{0.5\linewidth}{0.5pt}\end{center}

\subsubsection{Among the Usulis (Legal
Theorists)}\label{among-the-usulis-legal-theorists}

\textbf{al-Juwayni} (\emph{al-Burhan}), \textbf{al-Ghazali}
(\emph{al-Mustasfa}), \textbf{al-Razi} (\emph{al-Mahsool}), \textbf{Taj
al-Subki} (\emph{Jam' al-Jawami}), \textbf{al-Baydawi} (\emph{Minhaj
al-Usul})

\begin{center}\rule{0.5\linewidth}{0.5pt}\end{center}

\subsubsection{Among the Hadith
Scholars}\label{among-the-hadith-scholars}

\textbf{al-Daraqutni}, \textbf{Ibn Khuzaymah}, \textbf{Ibn Hibban},
\textbf{Abu Nu'aym}, \textbf{Ibn al-Mundhir}, \textbf{al-Khatabi},
\textbf{Khatib al-Baghdadi},\\
\textbf{al-Bayhaqi} (\emph{al-Sunan}), \textbf{al-Iraqi}
(\emph{Alfiyyah}), \textbf{al-Haythami} (\emph{Majma' al-Zawa'id}),
\textbf{Ibn Hajar al-Asqalani} (\emph{Fath al-Bari})

\begin{center}\rule{0.5\linewidth}{0.5pt}\end{center}

\subsubsection{Among the Historians and
Biographers}\label{among-the-historians-and-biographers}

\textbf{Ibn 'Asakir} (\emph{Tarikh Dimashq}), \textbf{al-Dhahabi}
(\emph{Siyar A'lam al-Nubala'}), \textbf{al-Safadi} (\emph{al-Wafi}),\\
\textbf{Ibn Kathir} (\emph{al-Bidayah wa al-Nihayah}), \textbf{Ibn
al-Athir} (\emph{al-Kamil})

\begin{center}\rule{0.5\linewidth}{0.5pt}\end{center}

\subsubsection{Among the Theologians
(Mutakallimeen)}\label{among-the-theologians-mutakallimeen}

\textbf{al-Halimi} (\emph{Shu'ab al-Iman}), \textbf{Abd al-Qahir
al-Baghdadi}, \textbf{Fakhr al-Din al-Razi} (\emph{al-Matalib
al-'Aliyah}),\\
\textbf{Adud al-Ayji}, \textbf{al-Aamidi}, \textbf{Ulaa al-Baji},
\textbf{al-Asfahani}, \textbf{al-Taftazani}

\begin{center}\rule{0.5\linewidth}{0.5pt}\end{center}

\subsubsection{Among the Quran
Commentators}\label{among-the-quran-commentators}

\textbf{al-Mawardi}, \textbf{al-Khazin}, \textbf{al-Baghawi}
(\emph{Ma'alim al-Tanzil})

\begin{center}\rule{0.5\linewidth}{0.5pt}\end{center}

\subsubsection{Among the Quran Reciters}\label{among-the-quran-reciters}

\textbf{al-Ja'bari}, \textbf{Ibn al-Jazari} (\emph{al-Nashr}),
\textbf{al-Qastallani}

\begin{center}\rule{0.5\linewidth}{0.5pt}\end{center}

\subsubsection{Among the Linguists and
Grammarians}\label{among-the-linguists-and-grammarians}

\textbf{Abu Hayyan al-Andalusi}, \textbf{Ibn Malik}
(\emph{al-Alfiyyah}), \textbf{Ibn Aqil}, \textbf{Ibn Hisham},
\textbf{al-Fayruzabadi} (\emph{al-Qamus})

\begin{center}\rule{0.5\linewidth}{0.5pt}\end{center}

\subsubsection{Among the Sufi Masters}\label{among-the-sufi-masters}

\textbf{al-Qushayri} (\emph{al-Risalah al-Qushayriyyah}),
\textbf{al-Ghazali}, \textbf{Abd Allah ibn Alawi al-Haddad}

\begin{center}\rule{0.5\linewidth}{0.5pt}\end{center}

And countless other imams and scholars who contributed to the sciences
and branches of knowledge across the Islamic tradition.

\subsection{A Summary of the History of the Shafi'i
Madhhab}\label{a-summary-of-the-history-of-the-shafii-madhhab}

The history of the madhhab can be summarized in five phases:

\subsubsection{1. The Founding Phase}\label{the-founding-phase}

This began with the establishment of the school and ended with the death
of Imam al-Shafi'i, may God be pleased with him. During this phase, he
authored foundational works such as \emph{al-Umm}.

\subsubsection{2. The Transmission Phase}\label{the-transmission-phase}

In this phase, his students and companions began spreading the madhhab.
Among the most well-known of their writings is \emph{Mukhtasar
al-Muzani}.

\begin{tcolorbox}[enhanced jigsaw, leftrule=.75mm, bottomtitle=1mm, colframe=quarto-callout-note-color-frame, colback=white, arc=.35mm, breakable, titlerule=0mm, opacitybacktitle=0.6, opacityback=0, colbacktitle=quarto-callout-note-color!10!white, toptitle=1mm, title=\textcolor{quarto-callout-note-color}{\faInfo}\hspace{0.5em}{The Legacy of al-Muzani and the Departure of al-Tahawi}, rightrule=.15mm, toprule=.15mm, left=2mm, coltitle=black, bottomrule=.15mm]

Al-Muzani, one of the closest students of Imam al-Shafi'i, played a
pivotal role in transmitting the madhhab. His \emph{Mukhtasar} became a
foundational reference for later scholars.

His nephew, Abu Ja'far al-Tahawi, initially studied under him in the
Shafi'i school. However, when al-Tahawi showed signs of leaning toward
the Hanafi madhhab, al-Muzani reportedly remarked:

\begin{quote}
``You will never attain a high rank in knowledge.''
\end{quote}

Feeling dismissed and underestimated, al-Tahawi left the Shafi'i school
and committed himself to the Hanafi tradition. In time, he rose to
become one of its most authoritative jurists---authoring works like
\emph{Sharh Ma'ani al-Athar} and \emph{Aqidah al-Tahawiyyah}, which
continue to be studied across madhhabs to this day.

\end{tcolorbox}

\subsubsection{3. The Expansion and Branching
Phase}\label{the-expansion-and-branching-phase}

This stage witnessed the documentation and expansion of legal issues
within the madhhab. Two major methodological approaches emerged:

\begin{itemize}
\item
  \textbf{The Iraqi Approach}: Led by Abu Hamid al-Isfarayini, followed
  by al-Mawardi, Abu al-Tayyib al-Tabari, al-Bandaniji, al-Mahamili,
  Sulaym al-Razi, and others.
\item
  \textbf{The Khurasani (Persian) Approach}: Led by al-Qaffal al-Saghir
  Abu Bakr al-Marwazi, followed by Abu Muhammad al-Juwayni, al-Furani,
  al-Qadi Husayn, Abu Ali al-Sinji, al-Masudi, and others.
\end{itemize}

\subsubsection{4. The Codification Phase}\label{the-codification-phase}

This stage was carried out at the hands of the two great scholars of the
madhhab: al-Rafi'i and al-Nawawi. Among the most important works
produced in this phase are:

\begin{itemize}
\tightlist
\item
  \emph{al-Muharrar} \emph{al-Sharh al-Saghir} and \emph{al-Sharh
  al-Kabir}, all by al-Rafi'i, which are themselves based on
  al-Ghazali's \emph{al-Wajiz}.
\item
  \emph{al-Minhaj}, \emph{al-Majmu' Sharh al-Muhadhdhab}, and
  \emph{Rawdat al-Talibin}, all by al-Nawawi, which likewise draw upon
  \emph{al-Wajiz}.
\end{itemize}

In these works, both scholars refined and systematically arranged the
issues of the madhhab, clarified the evidences, and reconciled between
the varying narrations and positions within the school.

\subsubsection{5. The Consolidation
Phase}\label{the-consolidation-phase}

This phase is marked by the efforts of two great scholars:

\begin{itemize}
\tightlist
\item
  \textbf{Ibn Hajar al-Haytami}: Author of \emph{Tuhfat al-Muhtaj bi
  Sharh al-Minhaj}\\
\item
  \textbf{al-Shams al-Ramli}: Author of \emph{Nihayat al-Muhtaj ila
  Sharh al-Minhaj}
\end{itemize}

In these works, both scholars compiled and analyzed rulings from across
the school. They clarified points left unaddressed by earlier jurists
(namely Nawawi and Rafi'i), referenced various transmitted positions,
and organized them into a coherent legal system covering the entire
breadth of jurisprudential topics.

\begin{tcolorbox}[enhanced jigsaw, leftrule=.75mm, bottomtitle=1mm, colframe=quarto-callout-note-color-frame, colback=white, arc=.35mm, breakable, titlerule=0mm, opacitybacktitle=0.6, opacityback=0, colbacktitle=quarto-callout-note-color!10!white, toptitle=1mm, title=\textcolor{quarto-callout-note-color}{\faInfo}\hspace{0.5em}{The Authority of Ibn Hajar and al-Ramli}, rightrule=.15mm, toprule=.15mm, left=2mm, coltitle=black, bottomrule=.15mm]

Once the madhhab had been firmly established through the works of
al-Nawawi and al-Rafi'i, it continued to flourish at the hands of Ibn
Hajar and al-Ramli. Later scholars accepted their writings as
authoritative in fatwa.

\begin{itemize}
\tightlist
\item
  When Nawawi and al-Rafi'i agree on a matter, their position is adopted
  as the \textbf{relied-upon} (mu'tamad).
\item
  If they (Nawawi and Rafi'i) differ, the \textbf{opinion of al-Nawawi}
  is preferred.
\item
  If al-Nawawi was silent on the matter, \textbf{either of the two} may
  be relied upon.
\item
  If they both disagreed on a matter not addressed by earlier scholars,
  \textbf{the people of the Hijaz and Hadramawt} prefer Ibn Hajar,
  whereas \textbf{Egyptians and Syrians} often favor al-Ramli.
\end{itemize}

\end{tcolorbox}

\subsection{On Other Authors and
Works}\label{on-other-authors-and-works}

As for other authors whose writings are often transmitted, their
opinions are accepted in legal practice and fatwa unless they are known
to contain error, carelessness, or weakness --- something only
recognized by those well-trained in this field, i.e one who has taken
from Shuyookh in this field.

\subsection{The Most Important Books of the Shafi'i
Madhhab}\label{the-most-important-books-of-the-shafii-madhhab}

Books of fiqh in the madhhab are generally divided into the following
categories: foundational texts (\emph{mutun}), commentaries
(\emph{shuruh}), supercommentaries (\emph{hawashi}), legal responses
(\emph{fatawa}), and others.

Among the most important books currently circulated in circles of
learning are:

\subsubsection{\texorpdfstring{Foundational Texts
(\emph{Mutun})}{Foundational Texts (Mutun)}}\label{foundational-texts-mutun}

\begin{itemize}
\tightlist
\item
  \emph{Al-Risalah al-Jami'ah} by \textbf{Ahmad ibn Zayn al-Habashi}
\item
  \emph{Safinat al-Najah} by \textbf{Salim ibn Sumayr al-Hadrami}
\item
  \emph{Al-Muqaddimah al-Hadramiyyah} (both large and small versions) by
  \textbf{Abd al-Rahman Ba Fadl}
\item
  \emph{Al-Yaqut al-Nafis} by \textbf{Ibn Ahmad ibn Umar al-Shatiri}
\item
  \emph{Matn al-Ghayah wal-Taqrib} by \textbf{Abu Shuja' al-Asfahani}
\item
  \emph{Safwat al-Zubad} by \textbf{Shihab al-Din Ahmad ibn Raslan}
\item
  \emph{Umdat al-Salik} by \textbf{Ibn Naqib}
\end{itemize}

\subsubsection{\texorpdfstring{Commentaries
(\emph{Shuruh})}{Commentaries (Shuruh)}}\label{commentaries-shuruh}

\begin{itemize}
\tightlist
\item
  \emph{Nayl al-Raja} --- on \emph{Safinat al-Najah} by \textbf{Ahmad
  ibn Umar al-Shatiri}
\item
  \emph{Bushra al-Karim} --- on \emph{al-Muqaddimah al-Hadramiyyah} by
  \textbf{Sa'id ibn Muhammad Ba'ishin}
\item
  Commentary on \emph{Matn Abi Shuja'} by \textbf{Ibn Qasim al-Ghazzi}
\item
  \emph{Al-Iqna'} by \textbf{al-Khatib al-Shirbini}
\item
  \emph{Fath al-Allam} --- on \emph{Murshid al-Anam} by
  \textbf{al-Jurdani}
\item
  \emph{Fath al-Wahhab} --- on \emph{al-Minhaj} by \textbf{Zakariya
  al-Ansari}
\item
  \emph{Fath al-Muin} --- on \emph{Qurrat al-Ayn} by \textbf{Zayn al-Din
  ibn Abd al-Aziz al-Malibari}
\end{itemize}

\paragraph{\texorpdfstring{Major Commentaries on Imam al-Nawawi's
\emph{Minhaj}}{Major Commentaries on Imam al-Nawawi's Minhaj}}\label{major-commentaries-on-imam-al-nawawis-minhaj}

\begin{itemize}
\tightlist
\item
  \emph{Kanz al-Raghibin} by \textbf{al-Muhalli}
\item
  \emph{Mughni al-Muhtaj} by \textbf{al-Khatib al-Shirbini}
\item
  \emph{Nihayat al-Muhtaj} by \textbf{al-Ramli}
\item
  \emph{Tuhfat al-Muhtaj} by \textbf{Ibn Hajar al-Haytami}
\end{itemize}

\begin{tcolorbox}[enhanced jigsaw, leftrule=.75mm, bottomtitle=1mm, colframe=quarto-callout-important-color-frame, colback=white, arc=.35mm, breakable, titlerule=0mm, opacitybacktitle=0.6, opacityback=0, colbacktitle=quarto-callout-important-color!10!white, toptitle=1mm, title=\textcolor{quarto-callout-important-color}{\faExclamation}\hspace{0.5em}{Why These Commentaries Matter}, rightrule=.15mm, toprule=.15mm, left=2mm, coltitle=black, bottomrule=.15mm]

These commentaries form the backbone of Shafi'i legal reasoning. They
are not merely explanatory --- they shape how the madhhab is
transmitted, applied in fatwa, and taught around the world.

\begin{itemize}
\tightlist
\item
  They clarify ambiguous expressions in the core texts (\emph{mutun}).
\item
  They reconcile variant opinions and serve as the primary references
  for jurists and students alike.
\item
  They preserve the reasoning of the madhhab's foremost authorities,
  including \textbf{Ibn Hajar}, \textbf{al-Ramli}, and
  \textbf{al-Nawawi}.
\end{itemize}

\subsubsection{Institutional Approaches}\label{institutional-approaches}

In institutions like \textbf{al-Azhar}, it is common for two or three
major commentaries to be read alongside the \emph{mutun} during lessons
with the teacher. This approach emphasizes layered analysis, discussion,
and legal evaluation in the classroom.

In contrast, in the \textbf{Hijaz}, particularly in \textbf{Yemen},
teachers focus primarily on the \emph{mutun} in class. Students are
expected to study the commentaries independently after mastering the
texts. This fosters precision, retention, and discipline in the early
stages of learning.

Each tradition carries its strengths --- al-Azhar cultivates comparative
analysis, while the Hijazi model emphasizes firm mastery before engaging
multiple views.

\end{tcolorbox}

\subsubsection{\texorpdfstring{Supercommentaries
(\emph{Hawashi})}{Supercommentaries (Hawashi)}}\label{supercommentaries-hawashi}

\begin{itemize}
\tightlist
\item
  Supercommentary by \textbf{al-Bajuri} on the commentary of \textbf{Ibn
  Qasim} on \emph{Matn Abi Shuja'}
\item
  Supercommentary by \textbf{Bakri ibn Shatta al-Dimyati} on \emph{Fath
  al-Muin}
\item
  Supercommentary by \textbf{Abd Allah ibn Hajjazi al-Sharqawi} on
  \emph{Sharh al-Tahrir}
\item
  Madinan tradition of supercommentaries on \textbf{Ibn Hajar}'s
  \emph{Muqaddimah}, notably:

  \begin{itemize}
  \tightlist
  \item
    \textbf{Muhammad ibn Sulayman al-Kurdi}
  \item
    \textbf{al-Tarmisi}
  \end{itemize}
\item
  \textbf{al-Bujayrimi} on \emph{al-Iqna'} of \textbf{al-Khatib}
\item
  \textbf{al-Jamal} and \textbf{al-Bujayrimi} on \emph{Sharh al-Minhaj}
\item
  Further supercommentaries on \emph{Minhaj al-Talibin} exist
\end{itemize}

\begin{tcolorbox}[enhanced jigsaw, leftrule=.75mm, bottomtitle=1mm, colframe=quarto-callout-note-color-frame, colback=white, arc=.35mm, breakable, titlerule=0mm, opacitybacktitle=0.6, opacityback=0, colbacktitle=quarto-callout-note-color!10!white, toptitle=1mm, title=\textcolor{quarto-callout-note-color}{\faInfo}\hspace{0.5em}{Unpublished Treasures}, rightrule=.15mm, toprule=.15mm, left=2mm, coltitle=black, bottomrule=.15mm]

Some supercommentaries remain in manuscript form and are in the process
of being published. Among them:

\begin{itemize}
\tightlist
\item
  \textbf{al-Dimyari} on various texts
\item
  \textbf{al-Azhari}
\item
  \textbf{Taqi al-Subki}
\end{itemize}

\end{tcolorbox}

Other important supercommentaries include:

\begin{itemize}
\tightlist
\item
  \textbf{Abd al-Hamid al-Sharwani} and \textbf{Ibn Qasim al-Ubadi} on
  \emph{Tuhfat al-Muhtaj}
\item
  \textbf{al-Qalubi} and \textbf{Umayrah} on \emph{Sharh al-Mahalli}
\item
  \textbf{al-Shirbini} and \textbf{al-Rashidi} on \emph{Nihayat
  al-Muhtaj}
\end{itemize}

\subsubsection{\texorpdfstring{Legal Opinions
(\emph{Fatawa})}{Legal Opinions (Fatawa)}}\label{legal-opinions-fatawa}

\begin{itemize}
\tightlist
\item
  \emph{Fatawa} of \textbf{al-Izz ibn Abd al-Salam}
\item
  \emph{Fatawa} of \textbf{al-Subki}
\item
  \emph{Fatawa al-Suyuti} (al-Hawi lil-Fatawi)
\item
  \emph{Fatawa Nawawiyyah} by \textbf{al-Nawawi}
\item
  \emph{Fatawa Kubra} by \textbf{Ibn Hajar}
\item
  \emph{Fatawa Bamakhramah}
\item
  \emph{Bughiyyat al-Mustarshidin} by \textbf{Abd al-Rahman al-Mashhur}
\end{itemize}

\subsubsection{Works on Takhrij
al-Hadith}\label{works-on-takhrij-al-hadith}

\begin{itemize}
\tightlist
\item
  \emph{Talkhis al-Habir} by \textbf{Ibn Hajar al-Asqalani}
\item
  \emph{Badr al-Munir} and \emph{Tuhfat al-Muhtaj} by \textbf{Ibn
  al-Mulaqqin}
\end{itemize}

\subsubsection{Books Supporting the Madhhab with
Evidences}\label{books-supporting-the-madhhab-with-evidences}

\begin{itemize}
\tightlist
\item
  \emph{Nihayat al-Matlab fi Adillat al-Madhhab} by \textbf{Imam
  al-Haramayn al-Juwayni}
\item
  \emph{al-Hawi al-Kabir} by \textbf{al-Mawardi}
\item
  \emph{al-Majmu'} by \textbf{al-Nawawi}
\item
  \emph{Fath al-Aziz} by \textbf{al-Rafi'i}
\item
  \emph{Sharh al-Minhaj} by \textbf{Taqi al-Din al-Subki}
\end{itemize}

\subsubsection{Fiqh Lexicons}\label{fiqh-lexicons}

\begin{itemize}
\tightlist
\item
  \emph{al-Misbah al-Munir} by \textbf{al-Fayyumi}
\item
  \emph{Tahrir al-Tanbih} and \emph{Daqaiq al-Minhaj} by
  \textbf{al-Nawawi}
\item
  \emph{al-Nazm al-Mustadhhab fi Hall Alfaz al-Madhhab} by \textbf{Ibn
  Batal al-Rukbi}
\end{itemize}

\subsubsection{Biographical
Dictionaries}\label{biographical-dictionaries}

\begin{itemize}
\tightlist
\item
  \emph{Tabaqat al-Shafi'iyyah} by \textbf{Ibn Asim al-Abbadi}
\item
  \emph{Tabaqat al-Shafi'iyyah al-Kubra} by \textbf{Taj al-Din al-Subki}
\item
  \emph{Tabaqat al-Shafi'iyyah} by \textbf{al-Isnawi}
\item
  \emph{Tabaqat al-Shafi'iyyah} by \textbf{Qadi Shuhbah}
\item
  \emph{al-Tuhfa al-Bahiyyah fi Tabaqat al-Shafi'iyyah} by
  \textbf{al-Sharqawi}
\end{itemize}

\subsection{Distinct Merits of the Shafi'i
Madhhab}\label{distinct-merits-of-the-shafii-madhhab}

The Shafi'i school is distinguished by many unique features, of which we
mention:

\subsubsection{1. Strong Foundation on Textual
Proof}\label{strong-foundation-on-textual-proof}

Its founder, may Allah have mercy on him, based the school firmly on
scriptural evidence from the Qur'an and Sunnah. He was a student of Imam
Malik and learned directly from the followers of the Prophet's
companions. Later scholars like Ahmad ibn Hanbal followed him, and the
great hadith masters such as al-Bayhaqi and Ibn Hajar al-Asqalani served
the madhhab by compiling evidences, demonstrating that many of the
leading hadith scholars (\emph{Huffadh, a memorizer of no less than
100,000 Ahadith with their respective chains of narraration!}) were from
the Shafi'i school.

\subsubsection{2. Groundbreaking Work in Legal
Theory}\label{groundbreaking-work-in-legal-theory}

Imam al-Shafi'i was the first to codify the principles of jurisprudence
(usul al-fiqh) in a structured work. His students and their students
developed major reference works, such as those by al-Juwayni and
al-Ghazali, that became foundational texts across madhhab lines.

\subsubsection{3. Balanced Methodology}\label{balanced-methodology}

The madhhab strikes a balance between the rationalist approach (ahl
al-ra'y) and the traditionalist approach (ahl al-hadith).

\subsubsection{4. Widespread Ijtihad}\label{widespread-ijtihad}

A large number of independent jurists (mujtahidun) emerged within the
madhhab and made contributions in every era and region, such as al-Izz
ibn Abd al-Salam, Ibn Daqiq al-'Id, al-Subki, al-Suyuti, and many
others.

\subsubsection{5. Rich Literature}\label{rich-literature}

There is an abundance of books written by scholars verifying, refining,
and supporting the madhhab throughout the centuries, making learning and
research easier for students.

\subsubsection{6. Global Reach}\label{global-reach}

Followers of the madhhab are found across the Muslim world --- from
Indonesia, Malaysia, and Southeast Asia to India, Persia, Iraq, the
Levant, the Arabian Gulf, Hijaz, Hadramawt, Yemen, Egypt, and even parts
of East Africa.

\subsubsection{7. Successive Revivers}\label{successive-revivers}

At the start of nearly every century, a notable reviver (mujaddid) of
the madhhab has emerged. Among them:

\begin{itemize}
\tightlist
\item
  Imam al-Shafi'i for the 2nd century\\
\item
  Abu al-Abbas ibn Surayj for the 3rd\\
\item
  Abu al-Tayyib Sahal al-Sa'luki for the 4th\\
\item
  Abu Hamid al-Ghazali for the 5th\\
\item
  Fakhr al-Din al-Razi for the 6th\\
\item
  Imam al-Nawawi for the 7th\\
\item
  Al-Isnawi for the 8th\\
\item
  Ibn Hajar al-Haytami for the 9th\\
\item
  Al-Suyuti for the 10th
\end{itemize}




\end{document}
